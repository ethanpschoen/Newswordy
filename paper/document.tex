\documentclass[10pt,twocolumn]{article}

% use the oxycomps style file
\usepackage{oxycomps}

% figures live in paper/figures/
\graphicspath{{figures/}}

% usage: \fixme[comments describing issue]{text to be fixed}
% define \fixme as not doing anything special
\newcommand{\fixme}[2][]{#2}
% overwrite it so it shows up as red
\renewcommand{\fixme}[2][]{\textcolor{red}{#2}}
% overwrite it again so related text shows as footnotes
%\renewcommand{\fixme}[2][]{\textcolor{red}{#2\footnote{#1}}}

% read references.bib for the bibtex data
\bibliography{references}

% include metadata in the generated pdf file
\pdfinfo{
    /Title (Newswordy: The Buzzword Guessing Game)
    /Author (Ethan Schoen)
}

% set the title and author information
\title{Newswordy: The Buzzword Guessing Game}
\author{Ethan Schoen}
\affiliation{Occidental College}
\email{eschoen@oxy.edu}

\begin{document}

\maketitle

\section{Introduction}

The contemporary news ecosystem is noisy, polarized, and difficult to navigate.
24-hour cable channels, online outlets, and social media platforms compete
for attention by emphasizing novelty, outrage, and conflict, often at the
expense of context and nuance\cite{JonesTheNew2012}.
This contributes to news fatigue, disengagement, and selective exposure,
where people either tune out entirely or restrict themselves to a narrow set of
ideologically aligned sources\cite{MartinBiasIn2017,FitzpatrickNoNews2022}.
At the same time, the proliferation of partisan media and misinformation makes
it harder for non-experts to develop a broad, trustworthy sense of what is
actually happening in the world\cite{KubinTheRole2021}.

When mainstream outlets are perceived as untrustworthy, people may turn to
more extreme or less rigorous sources, including influencer-driven commentary
and algorithmically curated social feeds\cite{SpohrFakeNews2017}.
In this environment, there are relatively few tools that help users compare
coverage across outlets at a glance or understand how different news sources
frame the same events.
Existing media bias charts and dashboards are valuable, but they tend to assume
a high level of prior motivation and literacy: users must already care enough
about news to seek them out, interpret visualizations, and read methodological
fine print\cite{HeldebrandtHowA2019}.

The problem this project addresses is how to provide an accessible, engaging
way for people to develop a broad awareness of current events and differences
in coverage across outlets, without requiring them to read large volumes of
articles or already be experts in media studies.
Rather than replacing traditional news consumption, the goal is to create an
\emph{on-ramp}: a low-friction daily habit that encourages curiosity about what
is in the news and how different sources talk about it.

\href{https://newswordy.vercel.app}{Newswordy (available at https://newswordy.vercel.app)}
is a web-based word-guessing game that operationalizes this idea.
Each day, it aggregates headlines from a curated set of major news outlets and
builds word-frequency summaries and comparisons over configurable time windows.
Players then try to guess the most frequent words in those headlines, earning
points based on how common their guesses are.
As they play, users can click on any word to see the underlying articles that
mention it, surfacing the context behind the ``buzzwords'' and inviting deeper
engagement with specific stories.

\begin{figure*}[t]
    \centering
    \includegraphics[width=0.49\textwidth]{homepage.png}
    \hfill
    \includegraphics[width=0.49\textwidth]{gameplay.png}
    \caption{Newswordy interface overview. Left: home page with mode selection.
    Right: gameplay view with guess input, scoreboard, and an article drawer for
    a selected buzzword.}
    \label{fig:ui-overview}
\end{figure*}

The game builds on the success of daily casual games such as Wordle,
crossword puzzles, and Duolingo, which have been shown to support regular engagement and incidental
learning\cite{LoewenMobileAssisted2019,RitcheyWordleAs2023,ZamaniTheUse2021}.
By framing news awareness as a playful, time-bounded challenge with clear
feedback, Newswordy aims to make it easier for both habitual news readers and
more casual consumers to track what topics dominate headlines and how that
differs across outlets.

This paper makes three main contributions.
First, it articulates the problem of news awareness and media comparison as an
interaction and motivation challenge, not only as an information retrieval
problem.
Second, it describes the design and implementation of Newswordy, including how
news data are collected, processed, and exposed through different game modes.
Third, it evaluates the system through qualitative user interviews, analyzing
to what extent the game is educational, engaging, and accessible, and
discusses limitations and ethical considerations of gamifying news consumption.

\section{Technical Background}

To understand the problem Newswordy addresses, it is useful to briefly review
how people typically encounter news today, and how this shapes polarization,
disengagement, and trust.

\subsection{Modern News Consumption}

Television, particularly cable news, remains one of the most influential
sources of political information for many adults.
Research has shown that cable outlets have increasingly diverged from each
other in both content and tone, contributing to partisan news diets and
ideological polarization\cite{MartinBiasIn2017}.
Programs often blur the boundary between hard news and opinion, prioritizing
emotional resonance and entertainment value to maintain viewership.

At the same time, social media platforms have become a major gateway to news,
especially for younger audiences.
Algorithmic feeds prioritize engagement, which tends to surface sensational,
outrage-inducing, or identity-affirming content\cite{KubinTheRole2021}.
Users encounter news interspersed with memes, personal updates, and advertising,
which can make it difficult to distinguish verified reporting from rumors or
manipulated narratives.

These dynamics contribute to two related challenges.
First, many people experience news as exhausting or demoralizing, leading to
selective avoidance and a sense that being informed is incompatible with their
mental health\cite{FitzpatrickNoNews2022}.
Second, those who do stay engaged often do so within relatively narrow
information ecosystems that reinforce their existing views.
Both outcomes undermine the civic goal of having a broadly informed public.

\subsection{News Headlines}

Headlines are a useful lens for understanding what ``the news'' is about because
they are the most visible, compressed representation of a story.
They encode what editors and platforms believe will attract attention, and they
often foreground the people, institutions, and frames that an outlet wants
readers to associate with an event.
While a headline cannot capture nuance, aggregating many headlines over time can
reveal consistent patterns in emphasis: which topics recur, which actors are
named, and which terms become salient across outlets.
For this reason, headlines provide a practical unit of analysis for lightweight,
comparative tools that aim to surface patterns in coverage without requiring
full-text article processing.

\section{Prior Work}

The project draws on three main strands of prior work:
tools for understanding media bias and news coverage, news quizzes and games,
and research on serious games and gamification for learning.

\subsection{Media Bias and Coverage Tools}

Several organizations provide visualizations and ratings intended to help
people reason about media bias and reliability.
Examples include news bias charts and sites that position outlets along
ideological and credibility axes based on content analysis and expert
judgments\cite{HeldebrandtHowA2019}.

These tools can be powerful for motivated users, but they have several
limitations relative to the problem Newswordy addresses.
They generally assume that users already care enough to seek out a specialized
site, interpret fairly dense visualizations, and read methodological notes.
They also tend to present bias as a static property of outlets rather than a
dynamic property of specific stories, time periods, or word choices.
By contrast, Newswordy emphasizes \emph{interaction}: instead of being told
that an outlet is left- or right-leaning, users can experiment with source
groups and time ranges to see how the vocabulary of coverage shifts.

\subsection{News Quizzes and Games}

Mainstream outlets frequently offer weekly ``news quizzes'' that test readers
on the details of recent stories\cite{ConillGamifyingThe2018}.
These quizzes can be engaging, but they typically require prior reading of the
same outlet's coverage and focus on factual recall rather than comparative
reasoning about media agendas.

In parallel, casual word games like crosswords and Wordle have become daily
rituals for millions of players.
Although they are not usually framed as news-literacy tools, they demonstrate
that low-friction, time-bounded puzzles can sustain long-term engagement and
support vocabulary and pattern learning\cite{RitcheyWordleAs2023}.

Newswordy combines these two lines of inspiration.
Like a news quiz, it is grounded in current events; like a daily puzzle, it
asks players to make educated guesses about words under light pressure.
However, instead of quizzing users on article-specific facts, it focuses on the
\emph{distribution} of words across headlines, encouraging players to think
about which topics dominate coverage and how that varies across sources.

\subsection{Serious Games and Gamification}

There is a rich literature on serious games --- games designed for purposes
such as education, training, or civic engagement\cite{FreitasAreGames2018}.
In practice, many of the most successful learning-oriented products borrow from
games without fully becoming games themselves.
Apps like Duolingo and Brilliant use short lessons, immediate feedback, and
visible progress to sustain engagement over long periods\cite{PuenteAssessingThe2023}. The use of streaks
is also a common feature of these games, intended to support habit formation.
These examples suggest that lightweight, repeatable interactions can support
learning goals when the mechanics align with the desired habit.
Research on gamification suggests that points, progress feedback, and social
comparison can increase short-term engagement, but that thoughtful alignment
between game mechanics and learning goals is crucial\cite{SailerTheGamification2023}.

\section{Methods}

\subsection{Design Goals}

The central design goal of Newswordy is to make high-level news awareness
feel approachable and even fun, while still respecting the seriousness of the
underlying topics \cite{FreitasAreGames2018}.
From this overarching goal, I derived several more concrete objectives:
\begin{itemize}
    \item \textbf{Broad awareness:} help players develop an at-a-glance sense
    of which topics dominate recent headlines.
    \item \textbf{Comparative insight:} make it easy to compare how different
    groups of outlets talk (or don't talk) about the same content.
    \item \textbf{Transparency:} always allow players to see the underlying
    articles behind aggregate statistics, and encourage further engagement
    with the content.
    \item \textbf{Low friction:} keep each play session short and intuitive,
    suitable for daily or weekly use.
\end{itemize}

\subsection{Data Collection}

To support these goals, the system collects headlines from a curated set of
more than fifteen major news sources, including both U.S. and international
outlets and a range of ideological perspectives\cite{KubinTheRole2021,MartinBiasIn2017}.
Each source is configured via its RSS feed, which serves as a proxy for its
``front page'' at any given time.

\begin{figure}[t]
    \centering
    \includegraphics[width=\columnwidth]{news-sources.png}
    \caption{Newswordy comparative mode source selection interface, which lets players choose
    curated outlet groups before starting a comparative game.}
    \label{fig:news-sources}
\end{figure}

A Python-based scraper runs twice daily via a scheduled workflow, fetching the
latest items from each feed.
For each article, the system stores metadata such as source, title, URL, and
publication timestamp in a PostgreSQL database.
Because RSS feeds can contain overlapping content across runs, the scraper
performs simple de-duplication based on URL and headline.

\subsection{Word Processing and Aggregation}

Word-level statistics are computed from the stored headlines.
For each article, the scraper lowercases the headline, tokenizes it into words,
and removes common stop words (e.g., ``the'', ``and'').
Remaining tokens are treated as candidate ``buzzwords.''

When a user starts a game, they may configure a time window (for
example, the past day or week) and a set of sources or source groups.
The backend then:
\begin{enumerate}
    \item Filters stored articles to those whose timestamps fall within the
    selected window and whose sources are in the chosen group(s).
    \item Counts occurrences of each token in the filtered set.
    \item Ranks tokens by frequency, producing a scoreboard of top words.
\end{enumerate}

For modes that involve comparison between two groups of sources, the system
determines the percentage of articles from each source mentioning a given word,
and finds the difference in percentages between the two groups.
For association-based modes, it first identifies all headlines containing the
chosen anchor word and then computes co-occurrence counts for other tokens
within those headlines.

These computations are implemented as database queries and functions so that
they can run close to the data and respond in real time as users start a game.
The algorithms themselves are simple by design, making it easier for players
to build an intuition for what the scores mean.

\subsection{Game Modes}

Newswordy exposes these statistics through four main game modes:
\begin{description}
    \item[Classic mode] shows a single scoreboard of the most frequent words
    for the selected time window and sources.
    Players guess words, earning more points for more frequent guesses.
    This mode emphasizes broad awareness of what is ``in the air.''
    \item[Comparison mode] splits sources into two configurable groups and asks
    players to guess words that are disproportionately frequent in one group.
    This highlights differences in coverage and agenda between outlets.
    \item[Association mode] lets players pick an anchor word and guess other
    words that commonly appear alongside it in headlines, surfacing the
    narrative context around specific topics.
    \item[Comparative association] combines the previous two ideas by comparing
    co-occurrence patterns for an anchor word across two source groups, helping
    players explore how different outlets talk about and frame the same topic.
\end{description}

\begin{figure}[t]
    \centering
    \includegraphics[width=\columnwidth]{game-modes.png}
    \caption{Game mode selection in Newswordy. The four modes correspond to
    (classic) overall word frequency, (comparison) differences between source
    groups, (association) co-occurrence around an anchor word, and (comparative
    association) co-occurrence differences between groups.}
    \label{fig:game-modes}
\end{figure}

All modes share a common interaction pattern: players type guesses into a text
field, the system checks whether the guess appears on the relevant scoreboard,
and correct guesses are revealed with their scores.
Hints are available to keep the experience approachable, including
reveals of the word's length and its first letter.

\subsection{Interface and Interaction Design}

The interface is designed to be readable on both desktop and mobile screens.
Key elements include:
\begin{itemize}
    \item A prominent scoreboard showing guessed and unguessed words.
    \item Controls for selecting time ranges and source presets.
    \item A persistent input field for guesses with immediate feedback.
    \item A sidebar or drawer that reveals the list of articles underlying any
    selected word.
\end{itemize}

Importantly, the article list appears as soon as a word is guessed and remains
available after the game ends, so players can always trace aggregate patterns
back to individual stories. Furthermore, this encourages players to explore the
content behind the words they guess, rather than simply thinking about the words
themselves. User accounts and profiles allow players to revisit past games, see
historical scores, and track their engagement over time, supporting habit formation.

\begin{figure}[t]
    \centering
    \includegraphics[width=\columnwidth]{profile.png}
    \caption{User profile view showing saved games and account-level statistics.}
    \label{fig:profile}
\end{figure}

\section{Evaluation Metrics}

Evaluating Newswordy requires assessing both its educational value and its
ability to engage users without overwhelming them \cite{PetriHowTo2016}.
Given the scope of this project, I focused on qualitative and self-reported
metrics obtained through user interviews, supplemented by observational notes
about how participants interacted with the interface.

\subsection{Research Questions}

The evaluation was organized around three main questions:
\begin{itemize}
    \item \textbf{Educational value:} Do players report feeling more aware of
    current events or more able to reason about differences in coverage after
    playing?
    \item \textbf{Support for critical comparison:} For users who already
    follow the news closely, do advanced modes (comparison and association)
    help them explore media bias and framing in new ways?
    \item \textbf{Engagement and usability:} Do players find the game engaging
    enough that they would choose to play it again?
    Is the interface intuitive enough that they would feel comfortable using it
    without guidance?
\end{itemize}

\subsection{Interview Protocol and Measures}

I conducted semi-structured interviews with six participants who varied in how
often they read the news and how confident they felt about their awareness of
current events.
The interview script (included in the repository as a CSV file) began by
asking about participants' baseline news habits (frequency, preferred sources,
and desired changes).

Participants then played a round of the classic mode while thinking aloud.
Afterwards, they answered questions about surprises, difficulty, and overall
impressions.
I then explained the advanced modes and invited them to choose one or more to
try.
Following this second round, participants reflected on whether the advanced
features enhanced their understanding or engagement.

Finally, I asked a series of summative questions:
what, if anything, they felt they had learned;
whether they would continue to play the game;
whether the interface felt intuitive enough to use independently; and what
improvements they would like to see.
These questions operationalize the three research questions above in terms of
self-reported learning, engagement intentions, and perceived usability.

\subsection{Alternative Metrics Considered}

In a longer-term study, it would be valuable to complement interviews with
quantitative measures, such as pre/post quizzes on current events, retention
tests administered days later, or detailed prompts about a specific finding \cite{PetriHowTo2016}.
However, these were beyond the scope of this project.

I also considered incorporating objective measures of exploration behavior,
such as the number of times participants opened article details or adjusted
source presets, as proxies for curiosity and critical comparison.
Due to time constraints and the small sample size, I relied instead on
participants' own reflections about how they used these features.

\section{Results}

The six interviews provide an initial picture of how different kinds of news
consumers experience Newswordy.
While the sample is small and not representative, several consistent themes
emerged around educational value, engagement, and the use of advanced modes.

\subsection{Baseline News Habits}

Most participants reported reading or watching the news every day, typically
through a mix of major outlets (such as the \emph{New York Times}, the
\emph{Washington Post}, and international sources) and social media.
A minority described themselves as less engaged and expressed either guilt
about not following the news closely or ambivalence about wanting to consume
less media for their own well-being.

Common motivations for change included wanting to read a wider variety of
sources and balancing a desire to be responsible and informed with the stress
of constant negative coverage.
These comments align with broader research on news avoidance and the emotional
burden of contemporary news consumption\cite{FitzpatrickNoNews2022}.

\subsection{Experiences with Classic Mode}

In the classic mode, participants generally found the core mechanic intuitive:
guessing words felt similar to other word games they had played.
Several noted that the game was harder than expected, largely because headline
vocabulary is unpredictable: proper nouns, niche terms, and event-specific names
can dominate a day, and the ``best'' guesses depend heavily on the selected time
window and source group.

Hints, especially partial word reveals, were frequently cited as helpful for
keeping the experience from becoming frustrating.
Participants were sometimes surprised that certain high-profile topics or
figures did \emph{not} appear among the top words, which led to informal
discussions about how headline writers choose what to foreground.

Importantly, even participants who already followed the news closely reported
new insights, such as noticing clusters of related words that highlighted
which aspects of ongoing stories were most salient in headlines.

\subsection{Use of Advanced Modes}

Participants who already engaged heavily with news tended to gravitate toward
the comparison and association modes.
They appreciated being able to see, for example, which words appeared more
often in left-leaning versus right-leaning outlets.
These interactions often led to discussion about why a given word might be
more common in one group, reflecting on both substantive differences in
coverage and stylistic choices.

A particularly notable instance was when users noticed that the left-leaning outlets
were more likely to talk about the Pope, which provoked larger discussion.
Users were surprised as they expected right-leaning outlets to be more likely to talk
about religion, but they hypothesized that since the Pope is considered more progressive,
the left-leaning sources would be more likely to talk about him and his left-leaning views.

Association-based modes also surfaced surprising connections.
Some participants remarked that the association lists made them more aware of
how certain topics are consistently linked to particular frames or co-occurring issues.
Others compared these patterns with what they saw on social media, noting that
social feeds sometimes emphasized different angles than mainstream outlets.

At the same time, a few participants tended to select topics they already felt
knowledgeable about, which limited the potential for learning new content.
This is a known challenge in educational games: players may optimize for
success rather than exploration\cite{SailerTheGamification2023}.

\subsection{Perceived Learning and Engagement}

Across interviews, participants reported that the game helped them feel more
aware of current events, even when they already followed the news.
For frequent news readers, the primary benefit was comparative:
advanced modes gave them a concrete way to inspect differences in coverage and
to validate or challenge their intuitions about media bias.

Participants who read the news less often emphasized that the game surfaced
stories they had not previously encountered and gave them a sense of ``what is
going on'' without requiring them to read full articles first.
One user remarked that the game was ``not only enjoyable, but also educational.''
Several said they would be likely to play on a weekly or monthly basis, or
that they could imagine playing competitively with friends or family.

Engagement was generally high during the sessions.
Colorful visual feedback, points, and visible progress all contributed to a
sense of accomplishment.
However, some participants also noted that the competitive framing sometimes
shifted their focus toward maximizing their score rather than reflecting on
the underlying news.
Designing future iterations to more explicitly reward exploration (for
example, by giving points for opening article details) could help rebalance
this trade-off.

\subsection{Usability and Interface Feedback}

Participants largely found the interface intuitive and stated that they would
feel comfortable navigating it without the researcher present.
They appreciated having presets for source groups and the ability to adjust
time periods.

Feedback highlighted several areas for improvement.
Some participants wanted the time-range selection to be more prominent, as
they did not initially realize they could adjust it.
There were also requests for more in-context explanation of certain advanced
features.

\subsection{Did the Game Solve the Problem?}

Within the limits of this small study, Newswordy appears to make progress on
the problem it sets out to address.
Participants reported that it helped them quickly grasp which topics
dominated recent headlines and, for advanced users, provided a concrete way to
interrogate differences in coverage between sources.
The ability to click through to underlying articles connects playful guessing
to deeper engagement with specific stories.

At the same time, the results highlight several caveats.
Because the sample is small and drawn from a relatively homogeneous population
of mostly college students, it is not clear how well the game would generalize to
other demographics.
Self-reported learning and engagement are also vulnerable to social desirability
bias.
Finally, the tendency of some participants to choose topics they already knew
well suggests that additional scaffolding might be needed to encourage
exploration of unfamiliar areas.

\section{Discussion}

The results suggest that Newswordy can serve as a lightweight ``on-ramp'' to
news awareness for less frequent readers while also offering a structured way
for more engaged readers to compare coverage across source groups.
Classic mode supports quick orientation to what dominated headlines in a given
window, and the advanced modes appear to be the main driver of reflective
conversation about framing and bias because they externalize differences as
inspectable word distributions linked back to concrete articles.

At the same time, the findings surface a central design tension in civic
gamification: the same score-and-feedback loop that makes play compelling can
pull attention away from interpretation.
Participants' comments suggest that future iterations should more explicitly
reward exploration (e.g., opening articles, switching source presets, or
following an association thread) and provide clearer scaffolding for how to
interpret comparative outputs without overclaiming what they mean.

This evaluation is intentionally preliminary.
Because the sample is small and relatively homogeneous, and outcomes are
self-reported, the results should be treated as suggestive evidence about user
experience rather than a definitive measure of learning.
A longer-term study could pair interviews with behavioral logging and
pre/post measures (e.g., recall, calibration about what is salient, or ability
to articulate multiple framings), and could test whether Newswordy changes
source diversity or reduces the friction of engaging with unfamiliar topics.

Overall, the project is most compelling as an interactive interface that makes
aggregate coverage patterns legible, contestable, and easy to trace back to
primary sources.
Newswordy is not meant to replace reading; it is meant to lower the barrier to
asking better questions about what is being covered, by whom, and with what
framing.
By grounding every pattern in transparent provenance (the underlying headlines
and articles), it invites users to move from ``what are the top words?'' to
``why does this show up here, and what does it obscure?''---turning passive
consumption into a repeatable practice of comparison and inquiry.

\section{Ethical Considerations}

Any system that summarizes news coverage and presents it as a game raises
ethical questions about bias, representation, and the potential trivialization
of serious events.
In designing Newswordy, I grappled with these issues in several ways.

\subsection{Source Selection and Representation}

The choice of which outlets to include, and how to group them, inevitably
reflects normative judgments about which sources are credible and worth
comparing.
Although I aimed for a mix of U.S. and international outlets and a range of
political orientations, the selection is still skewed toward large,
English-language organizations.
This can reinforce an existing hierarchy of whose perspectives count as
``mainstream'' and whose are peripheral\cite{HeldebrandtHowA2019}.

Moreover, focusing on headlines means inheriting all of the biases and blind
spots of professional newsrooms.
If certain communities or issues are consistently undercovered, Newswordy will
not surface them either.
The game makes these patterns more visible, but it does not by itself correct
for structural inequities in news production.

\subsection{Gamification and Trivialization}

Turning news into a word-guessing game risks trivializing topics that are
deeply consequential for people's lives.
Points, streaks, and competitive comparisons can encourage players to treat
headlines as abstract tokens rather than stories about real events.
This tension is not unique to Newswordy; it reflects broader concerns about
gamification in civic and political contexts\cite{FreitasAreGames2018}.

To mitigate this, the design emphasizes transparency and context.
Every aggregate statistic is backed by a list of concrete articles that
players can read.
There is no attempt to label outlets as ``good'' or ``bad''; instead, the
interface invites users to explore and draw their own conclusions.
Nevertheless, designers and researchers should remain alert to the possibility
that some players will engage primarily with the competitive aspects and
disengage from the underlying content.

\subsection{Interpretation of Comparative Metrics}

Comparison modes can be powerful but also easy to misinterpret.
If a word appears more frequently in one group of sources than another, that
does not necessarily mean that one side ``cares more'' about a topic; it may
reflect differences in geographic focus, audience, or editorial style.
Without careful explanation, players might overgeneralize from patterns in the
data to broad claims about media bias or ideological intent.

To address this, the interface encourages players to inspect the underlying
headlines rather than relying solely on aggregate scores, and the paper itself
frames results as exploratory rather than definitive.
Future work could incorporate explicit in-game disclaimers or walkthroughs
that demonstrate common pitfalls in interpreting comparative statistics.

\subsection{Data and Privacy Considerations}

From a data perspective, Newswordy minimizes risk by relying primarily on
publicly available RSS feeds and by storing only limited information about
users (such as anonymous identifiers and gameplay statistics).
However, any system that tracks user behavior must still consider issues of
consent, retention, and potential misuse.
If the project were deployed at larger scale, more explicit privacy policies
and options for data export or deletion would be necessary.

\section{Replication Instructions}

This section outlines how another computer science student could set up and
run Newswordy using the public GitHub repository, available at
\url{https://github.com/ethanpschoen/newswordy}.
More detailed, continuously updated instructions are provided in the
repository's README.

\subsection{Software and Accounts}

To replicate the project, a developer will need:
\begin{itemize}
    \item A recent version of Node.js and \texttt{npm} for the web frontend.
    \item Python 3.11 or later for the scraping pipeline.
    \item A PostgreSQL-compatible database, such as a Supabase project.
    When using Supabase, the frontend can connect via the Supabase client using
    the project's public URL and anon key, while database-side SQL functions can
    expose higher-level queries (e.g., ``top words'') through RPC calls.
    \item Accounts with a hosting provider (e.g., Vercel) and an authentication
    provider (e.g., Auth0) if they wish to deploy the app publicly.
\end{itemize}

\subsection{Scraper and Database Setup}

To populate the database with headlines, the developer can:
\begin{enumerate}
    \item \texttt{cd scraper}
    \item Create and activate a Python virtual environment.
    \item \texttt{pip install -r requirements.txt} to install scraping and
    text-processing dependencies.
    \item Configure database connection parameters via a \texttt{.env} file,
    using the example provided in the repository.
    \item Run the database initialization script to create necessary tables.
    \item Run the scraper manually to fetch and process headlines.
    \item Define the Row-Level Security (RLS) rules for the database tables.
    \item Add the necessary database functions to support the game modes.
\end{enumerate}

For automated updates, the repository includes a scheduled workflow that runs
the scraper twice daily.
To enable this, the developer must configure the appropriate database
credentials as secrets in their GitHub repository.

\subsection{Frontend Setup}

After cloning the repository, the developer can navigate to the frontend
directory, install dependencies, and start a development server:
\begin{enumerate}
    \item \texttt{cd frontend}
    \item \texttt{npm install}
    \item Create a \texttt{.env} file with the appropriate Supabase and Auth0
    keys, following the example file in the repository.
    \item \texttt{npm start} to launch the app locally.
\end{enumerate}

The frontend communicates directly with the database via a client library, so
no separate backend API server is required.

\subsection{Reproducibility Considerations}

Because external APIs and news content can change over time, exact numerical
results may not be reproducible across deployments.
However, by fixing software versions (for example, via \texttt{requirements.txt}
and lock files) and using the same set of news sources, another developer
should be able to recreate the overall system behavior and conduct their own
evaluations.

\section{Code Architecture Overview}

At a high level, Newswordy is organized into three main components: the news
scraping and processing pipeline, the database layer that stores articles and
supports word-frequency queries, and the web frontend.

\subsection{Scraper and Processing Pipeline}

The scraping code is organized around a configuration file that lists supported
news sources and their RSS endpoints.
Generic scraping utilities handle fetching feeds, parsing entries, and
normalizing fields such as timestamps.
Word processing utilities are responsible for tokenizing headlines, removing
stop words, and preparing data for storage.

Database interaction code defines models or helper functions for inserting and
querying articles and word statistics.
This separation of concerns allows developers to extend the system in several
ways, such as adding new sources, adjusting tokenization rules, or storing
additional metadata (for example, topic labels) without rewriting core logic.

If a developer wishes to add new sources, they would need to add the source
to the configuration file (alongside its RSS feed) and then run the scraper
to populate the database. This would require a corresponding addition to the
frontend with the name and logo of the source.

Potential extensions include changing the data extraction source from RSS feeds
or improving normalization so that related forms of a word (e.g., pluralization)
are treated consistently.
More ambitious extensions could incorporate lemmatization, phrase extraction
(bigrams), or topic labels, enabling game modes that surface themes rather than
only single tokens.

\subsection{Database and Query Layer}

The database schema distinguishes between raw articles and derived statistics.
Tables store articles with fields for source, title, URL, and publication
time, while additional tables or views support aggregated word counts.

Query functions encapsulate the logic for computing top words, comparisons
between source groups, and association patterns for anchor words.
Because these functions live in the database, the frontend can request
high-level operations (such as ``top words for this source group and time
range'') without needing to implement complex logic in client code.

To support player accounts and longitudinal statistics, the schema also
includes tables that record gameplay and user statistics.
For each game mode, there are two corresponding tables: one for the game itself
and one for the guesses made in that game.

When a player makes a new game, a new game record (and its ID) is created,
and as they make guesses, the guesses are recorded in
the guesses table. The game record is updated with each change in state,
and the guesses table is updated with the guesses and their scores.
This was done so that the frontend could easily display the game state and guesses,
and to support features like streaks, personal history pages, and aggregate analysis
of which words users tend to miss.

The game ID is used on the frontend to match the game settings from the database,
which is then used to fetch the list of top words for the game, using the RPC functions.

\subsection{Frontend Structure}

The frontend is implemented as a single-page application with routes
corresponding to different game modes and profile views.
Reusable UI components handle common elements such as the scoreboard, guess
input, hints, article information drawers, and game statistics.
Service modules encapsulate communication with the database and authentication
provider, so that changes to backend APIs or credentials can be made in one
place.

This modular structure makes it straightforward to add new game modes or
variants: a developer can create a new page that composes existing components
and issues queries against new or existing database functions.

To add new pages, a developer can add the route to \texttt{App.tsx} and create
the new page component in the \texttt{pages} directory. New game modes can build
off of the existing components and structure.

Potential frontend extensions include a leaderboard, new pages that explain
how to interpret comparisons, richer article browsing and filtering, and
additional game modes that focus on specific beats (e.g., politics-only sources)
or new regions and languages as additional headline sources are integrated.

\bigskip

Overall, this architecture is intended to make the system both understandable
to new contributors and extensible to new research questions about news
coverage and media bias.

\printbibliography

\end{document}
